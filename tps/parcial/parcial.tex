\documentclass[parcial]{lcc}

% add latex preamble
\input{../../common/latex_preamble}

% add math preamble
\input{../../common/math_preamble}

\codigo{R-521}
\materia{Robótica Móvil}
\titulo{Parcial}

\soluciones
\commentstrue


\usepackage{biblatex}
%\addbibresource{refs.bib}

\begin{document}
\maketitle

%section{Sensores}
%\ejercicio Dibuje el árbol de transformaciones


%\ejercicio Explique el comportamiento de la matriz de covarianza en las etapas de predicción y corrección de EKF?

\ejercicio In the Practical Assignment, where and why did you apply the \lstinline[style=python]{utils.minimized\_angle()} function ?

%\ejercicio What is the meaning of the covariance matrix?

\ejercicio Explain the common behaviour of the covariance matrix in the prediction step and the correction step of the EKF?

\ejercicio What are the main difference and similarities between PF and EKF?

%\ejercicio In PF, What could happen if all the particles have the same weight?

%\ejercicio In PF, What can happen if there used a too low or a too high number of particle?

%\ejercicio What happen if a loop closing is done in an EKF-SLAM?

%\ejercicio What are the two types of Graph-SLAM techniques that we have?

\ejercicio Why are the main source errors for a SLAM system?

\ejercicio What are the main difference between EKF-SLAM and Graph-SLAM?


\ejercicio 

\begin{enumerate}

    \item Consider the dynamical system described by the following model:

    \begin{equation*}
    x_{k+1} = \frac{1}{2}x_{k} \qquad x_{k} \in \mathbb{R}
    \end{equation*}
    The model is considered perfect.

    The initial value of the variable $x_0$ is not perfectly known, that is, there is only a belief about $x_0$. This belief is Gaussian with expected value 2 and variance $\sigma^2_{0} = 128$. (Note: the units used for both parameters are consistent.)

    You are required to provide predictions for $x_k$ for the discrete times $k=[1,2,3,4]$.

    Each predicted value must be provided as an expected value and associated variance, $\left(\mu_{k}, \sigma_{k}^2\right)$.

    \item Consider the same system mentioned in item (a), but now the model is considered to be imperfect. The new model is expressed as follows:
    \begin{equation*}
    x_{k+1} = \frac{1}{2}x_k + \eta_k
    \end{equation*}
    where the component $\eta_k$ is Gaussian white noise with an expected value of 0 and variance of 1.

    You are required to provide predictions for $x_{k}$, for discrete time $k=2$. Your belief about $x_0$ is the same as that considered in case (a).

\end{enumerate}

Note: You can express the results using expressions, including fractions, multiplications, etc. It is NOT strictly necessary to evaluate them, unless they are simple. For example, an expression such as $(1/3 + 2^3)*5$ can be expressed like this without being evaluated.

\ejercicio Consider a system whose state is represented, at discrete time k, by the vector
    \begin{equation*}
    X_{k} = \begin{bmatrix}
    x_{k}\\
    y_{k}
    \end{bmatrix}
    \end{equation*}

    Suppose that an estimate of the state (at time $k = 1$) is given by

    \begin{equation*}
    \mu_{1} = \begin{bmatrix} 1 \\ 2 \end{bmatrix}, \quad \covariance_{1} = \begin{bmatrix} 1 & 1 \\ 1 & 2 \end{bmatrix}
    \end{equation*}

    Suppose that at the same instant $k=1$ a certain output of the system is measured, whose functional relationship with the state is

    \begin{equation*}
    h\left(X_{k}\right) = \left(x_{k}\right)^3 + y_{k}
    \end{equation*}

    The measurement is taken through a sensor; the reading is $y_{1} = 3$. The measurement is known to be contaminated with Gaussian white noise, whose standard deviation is $0.5$ (assume the units are consistent). Based on this measurement, you perform an EKF (Extended Kalman Filter) update on the current estimates. One way to do this is through the following operations: 

    \begin{algorithmic}[1] 
    \State $S_{k} = \observationModelJacobian_{k} \overline{\covariance}_{k} \observationModelJacobian_{k}^{\top} + \observationModelCovariance_{k} $ 

    \State $\kalmanGain_{k} = \overline{\covariance}_{k} {\observationModelJacobian_{k}}^{\top} S_{k}^{-1} $ 
    \State $\mu_{k} = \overline{\mu}_{k} + \kalmanGain_{k}(\observation_{k} - h(\overline{\mu}_{k}))$ 
    \State $\covariance_{k} = (I - \kalmanGain_{k}\observationModelJacobian_{k})\overline{\covariance}_{k}$
    \end{algorithmic}

    \begin{enumerate}
    \item Use the provided expressions to implement the EKF update. E.g., specify the values that the following elements would take: $\observationModelJacobian_{k}$, $\observationModelCovariance_{k}$, $\overline{\mu}_{k}$, $\overline{\covariance}_{k}$.
    \item State what the following elements mean (represent): $\overline{\mu}_{k}$, $\overline{\covariance}_{k}$, $\mu_{k}$, $\covariance_{k}$.
    \end{enumerate}
\end{document}
